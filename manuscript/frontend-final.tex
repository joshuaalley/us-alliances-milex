
%``Moreover, even under the rosiest scenarios, curtailing America’s alliance military com- mitments would save something in the vicin- ity of 1 percent of gdp''
%https://www.jstor.org/stable/26557320?casa_token=ej5vXAXtQx4AAAAA%3Ak3OHJPMegfFA6ZFOZwEVMasbLm_RSDLUU5dukz1LNhLRmyUOZKOMhA0GUyEk783AZtYmsdRQWFbuK3J85j9Qqua6xsWSl3LE0bWzZQv9cqTV5N4Bcw&seq=1#metadata_info_tab_contents

%If the United States could achieve its military policy goals, other than pro-moting global stability, with a $150 billion defense budget—that is, if the United States only requires the rest of the force to maintain the alliances and sustain the operations that enhance stability—then the budgetary cost of the stability mission is $150 billion.
%https://www.tandfonline.com/doi/pdf/10.1080/09636410108429444?casa_token=sfy6T7MDJrAAAAAA:d0hHmbbUY7Gol5zZJq55E9kwPVkcPPLC3UYtI1jzeNxDDIrrRsXjtwitRDGXGNqCcW88tWsfUoE , p. 54
 
%Posen: ``I have argued that if the United States were more judicious in its promises abroad, perhaps a fifth of the defense budget could be cut (excluding the costs of actual wars), amounting to roughly one hundred billion dollars per year at current prices. ''

%Some figures on the costs of overseas troop deployments: https://theconversation.com/why-does-the-us-pay-so-much-for-the-defense-of-its-allies-5-questions-answered-127683

In his farewell address, which appeared in print on September 19, 1796, George Washington warned the United States not to form permanent alliances.\footnote{The United States entered into one defense pact during its revolutionary period before Washington resumed his presidency: a 1778 alliance with France, which officially ended in 1800.} U.S. leaders heeded this advice for more than 150 years. After World War II, however, the United States entered into a flurry of formal defense pacts in which it agreed to protect other countries from attack. It formed multilateral arrangements such as the Inter-American Treaty of Reciprocal Assistance (Rio Pact), the North Atlantic Treaty Organization (NATO), the Southeast Asia Treaty Organization (SEATO), and the Australia, New Zealand, United States Security Treaty (ANZUS). In addition, in the early days of the Cold War, Washington gave bilateral security assurances to Iran, Japan, South Korea, Taiwan, and the Philippines. American defense commitments continue to expand. Most recently, on March 27, 2020, the United States put North Macedonia under its security umbrella. In total, nearly 70 countries have treaty-backed American protection today.

%This article investigates the financial implications of forming and maintaining alliances for the United States. How do alliance commitments alter the U.S. defense budget? The general puzzle of whether alliances increase or decrease military spending\footcite[See, for example,][]{DigiuseppePoast2016,Diehl1994} is especially important in the U.S. context, as disagreements about the budgetary burden of alliances figure prominently in scholarship on U.S. foreign policy and grand strategy as well as in contemporary policy debates. 

Many studies have analyzed the implications of U.S. military alliances for international peace and conflict.\footcite[Much of this work includes, but is not limited to, U.S. alliance commitments. See, for example,][]{leedsAJPS03,FuhrmannSechser2014,aveyJGSS18,mcmanusJOP18} This article investigates a consequence of security guarantees that has been relatively overlooked in scholarship on alliance politics: the financial cost of forming and maintaining defense commitments for the United States. How do alliance commitments alter the U.S. defense budget? The answer can provide a starting point for evaluating U.S. support for military alliances in the Biden Administration and beyond. Knowing the degree to which alliances benefit the United States depends on having a more systematic sense of their budgetary effects.\footnote{As we discuss in the conclusion, whether alliances ultimately benefit the United States depends on a net assessment that incorporates the many political and economic benefits of security guarantees, not just the costs.}
%The general puzzle of whether alliances increase or decrease military spending\autocite[See, for example,][]{DigiuseppePoast2016,Diehl1994} is especially important in the U.S. context, as disagreements about the budgetary burden of alliances figure prominently in debates about U.S. foreign policy and grand strategy. 

We develop and test two competing theories on the budgetary implications of U.S. alliances.\footnote{These two perspectives represent ideal types. Instead of falling neatly into one of these camps, scholars may accept some tenets of the budget hawk view and others from the bargain hunter school.} One line of thinking, which we call the \textit{budget hawk} school, claims that alliances require large increases in defense spending. A \textit{bargain hunter} perspective, by contrast, maintains that U.S. alliance commitments are relatively cheap and may actually reduce the defense budget.  
%This is partially because alliances necessitate investments in technology, overseas bases, and costly coordination measures. On top of this, the United States must spend more to compensate for ``free-riding'' by allies who do not contribute their fair share and bail out partners who behave recklessly. A \textit{bargain hunting} perspective, by contrast, maintains that U.S. alliance commitments are relatively cheap and may actually reduce the defense budget.  In this view, alliances generate efficiencies that free up U.S. resources and do not necessarily depend on expensive ``sunk-cost'' signals, like troop deployments. And alliances promote peace by reining-in allies that might otherwise behave aggressively and deterring third-party aggression, saving the United States from fighting costly military conflicts. 

Knowing which view is correct carries implications for scholarly debate about whether alliances increase or decrease military spending.\autocite[See, for example,][]{Alley2020,DigiuseppePoast2016,Diehl1994} Adjudicating between the budget hawk and bargain hunter perspectives will also speak to a broader debate about the United States' role in the world. Arguments in favor of a more restrained foreign policy hinge heavily on the notion that alliance commitments necessitate large increases in U.S. defense spending.\autocite[See, for example,][]{gholzIS97,layneIS97,Posen2014} John Mearsheimer and Stephen Walt begin their ``bottom line'' case for a restrained grand strategy by claiming that it ``would allow the United States to markedly reduce its defense spending.''\autocite[83]{mearsheimerFA16} Stephen Brooks and William Wohlforth, who argue for continued deep U.S. engagement in world politics, aptly characterize the pervasiveness of this view in the restraint camp: ``Almost no call for retrenchment is complete without invoking the budgetary costs of deep engagement, which is arguably the most prominent argument for pulling back in the wider public debate about U.S. foreign policy.''\autocite[123]{brooksamerica16} Evidence that U.S. alliances decrease (or only modestly increase) defense spending would therefore substantially weaken the case for restraint, while a large positive effect of alliances on expenditures would strengthen it.
%By contrast, a large positive effect of alliances on U.S. military spending would strengthen the case for restraint. If such a relationship holds, proponents of deep engagement would have to demonstrate that the political and economic benefits of alliances exceed the military spending costs.%, not just argue that the budgetary effects are overstated by the restraint camp.
%because ``withdrawals from Europe and the Persian Gulf would free up billions of dollars''

%Which of these views is correct? The answer has implications for a broader debate about the United States' role in the world. 

%'Stephen Brooks, John Ikenberry, and William Wohlforth call the decision to extend security guarantees ``the United States' most consequential strategic choice.''\autocite[11]{Brooksetal2013} This choice has paid off for the United States in many ways. Studies show that alliances promote U.S. interests by reducing the risk of military conflict, limiting the spread of nuclear weapons, and increasing trade.\footnote{See, for example, \autocite{leedsAJPS03,JohnsonLeeds2011,FuhrmannSechser2014,bleekJCR14,reiterFPA14,Gowa:1993aa,longJPR03,liJIBS10}} Determining whether a foreign policy is desirable, however, also requires an assessment of its costs. Knowing how alliances affect the defense budget  allows us to more completely assess the benefits the United States reaps from proffering security guarantees relative to the size of its financial investment. 

Despite the importance of this question, the true spending costs of U.S. alliance commitments remain unclear. Few studies have attempted to systematically address this issue. As Mira Rapp-Hooper observes, arguments about the budgetary implications of U.S. alliance commitments ``are verifiable, yet, with a few exceptions, scholars have demurred.''\autocite[4]{rapphoopershields20} 

A large literature in political science and economics scrutinizes U.S. defense spending.\autocite[See, for example,][]{mintzpolitical92,fordhamJOP02,zielinskiSS19} These studies call attention to a diverse set of explanatory factors such as political parties, unemployment, bureaucratic politics, path dependence, and military conflict. However, U.S. alliance commitments play little role in this body of work. Political scientists and economists have also published many books and articles about the general relationship between alliances and defense spending.\autocite[See, for example,][]{OlsonZeckhauser1966,Sandler1993,Palmer1990,Fuhrmann2020} Virtually none of this work seeks to determine how changes in alliance commitments affect U.S. defense spending.\footnote{Paul Diehl distinguishes between major and non-major powers but does not separate the United States from countries such as France and the United Kingdom -- two states that many argue free-ride on Washington to some degree. See \cite{Diehl1994}} The research objective in these studies is instead to identify the average effect of alliances on military spending across a broad range of countries. There are good reasons to think that the United States might deviate from the general trend, however, because it is the strongest member of every defense pact it has forged since 1945. We therefore need theories and analysis that are specific to the U.S. context, which this article provides.
%\footnote{Some of the literature that takes a cross-national approach incorporates information on all (or most) military alliances. Other studies focus on specific alliances to which the United States is party, like NATO, but emphasize whether weaker alliance partners free-ride on the United States.}

U.S. grand strategy has been relatively constant since 1945.\footnote{\cite[12]{Brooksetal2013}. The biggest changes arguably occurred when Donald Trump was president.} However, the number of U.S. alliance commitments has fluctuated considerably during this time as Washington took on new commitments and abandoned some old ones. Membership also varies within the same alliance. To draw on the best-known example, NATO expanded eight times, going from 12 members in 1949 to 30 today. Such variation in the number of U.S. alliance commitments over time allows us to estimate how security guarantees influence U.S. defense expenditures. 
%To cite another example, Bolivia, Cuba, Ecuador, and Nicaragua withdrew from the Rio Pact in 2012. 

We use a dynamic regression model to estimate how changes in alliance commitments from 1945 to 2019 influence the U.S. defense budget. Using this statistical model, we calculate the long-run budgetary impact of alliance commitments. This is important because the costs and savings of alliances accrue over many years, not just immediately after alliance formation. Our model also controls for the president's party, war intensity, and other confounding factors. 

We find that the number of countries with U.S. protection is positively correlated with defense expenditures. Making one additional commitment increases the size of the U.S. defense budget by between \$11 and \$21 billion.\footnote{Our most conservative estimate from a series of robustness checks suggests that a new alliance adds between \$6 and \$15 billion to the defense budget on average.} This is a large association considering that the total U.S. defense budget in 2019 was \$686 billion.\autocite{collinsDO20} A series of robustness tests, including the use of presidential fixed effects, consistently find a large positive association between new alliance commitments and defense spending. 

These results support the budget hawk view and contradict the bargain hunter school. Many countries save money by forming military alliances.\autocite[See, for example,][]{DigiuseppePoast2016} However, based on our analysis, the United States is not one of them. Over the last 75 years, taking on an additional alliance commitment has, on average, considerably expanded the U.S. defense budget. This evidence bolsters a key tenet of the restraint approach to grand strategy. But it does not necessarily imply that the United States should terminate or rollback its existing alliance commitments, as many restrainers advocate. To know whether alliances are worth it for Washington, we must weigh the gains against the budgetary burden, a point that we return to in the paper's final section.

%Given that the financial costs are substantial, the political and economic benefits must be large for security guarantees to be desirable from a U.S. standpoint. If they are, extending security guarantees could still benefit the United States, despite their financial cost. Now that we have a reliable estimate on the cost side of the ledger, future research can more effectively assess the net benefits of U.S. alliance commitments. 

This article proceeds in five sections. First, we describe the logic of the budget hawk school. Second, we lay out the arguments behind the bargain hunter perspective. Third, we describe our research strategy for estimating the budgetary effects of U.S. alliance commitments. Fourth, we present our findings. The fifth section concludes by discussing the implications of our results for U.S. grand strategy and foreign policy, as well as highlighting the limitations of our study.


\section*{Budget Hawks: U.S. Alliances Increase the Financial Burden}


Many scholars have argued that U.S. alliances lead to massive defense spending increases.\autocite{gholzSS01,layneCRIA11,friedmanO12} U.S. presidents have similarly emphasized the cost of alliances to U.S. taxpayers. In January 1963, for example, John F. Kennedy privately complained to his advisers that the NATO allies were living off the ``fat of the land'' while his government spent large sums protecting Europe.\autocite{fruskennedy, creswellWOTR17} Budget hawks make a variety of arguments about the financial burden of U.S. alliance commitments. Our goal here is to lay out the full set of reasons why alliances might increase U.S. defense expenditures, not to characterize any single piece of scholarship.
%Christopher Layne, for example, writes that U.S. ``strategic commitments exceed the resources available to support them.''\autocite{layneCRIA11} Benjamin Friedman and Justin Logan echo this view, arguing that alliance commitments may be more costly than war -- the thing defense pacts are designed to prevent.\footnote{\cite[182]{friedmanO12}. See also \cite{gholzSS01}.} Many U.S. presidents have emphasized the cost of alliances to U.S. taxpayers. In January 1963, for example, John F. Kennedy privately complained to his advisers that the NATO allies were living off the ``fat of the land'' while his government spent large sums of money to protect Europe.\autocite{fruskennedy, creswellWOTR17}

%Trump has repeatedly made statements along these lines. As he said in March 2016, ``NATO is costing us a fortune, and yes, we're protecting Europe with NATO, but we're spending a lot of money.''\autocite[Quoted in][]{ruckerWP20160321} 

%Three main claims underlie this perspective: protecting other states requires expensive military investments; allies free-ride on the United States, forcing Washington to spend more to make up the difference; and allies behave recklessly, necessitating costly U.S. interventions to bail them out. Not everyone from the budget hawk school makes -- or necessarily agrees with -- all of these arguments. 

\subsection*{Direct Costs of Protection}

Defense pacts seek to deter third-party attacks and reassure allies that they can rely on partners for security. Countries must do a minimum of two things to make their alliance promises credible in the eyes of allies and potential attackers.\autocite[On the issue of credibility in alliance politics, see, for example,][]{schellingarms66,Fearon1997,Morrow2000} They must first have enough military capability to defend the ally. Yet capabilities alone are insufficient for success. The country offering protection must also convince other states that it would, in fact, intervene to fight on behalf of its ally in the event of war. 

Establishing willingness to intervene is particularly challenging. Allies and potential aggressors have frequently questioned the credibility of an alliance promise, even when the protector has the capacity to fulfill it. For example, French leader Charles de Gaulle knew that the United States \textit{could} defend France from a Soviet attack during the Cold War but doubted that Washington had the \textit{political will} to defend another country more than 3,000 miles away. As a result, De Gaulle developed an independent nuclear arsenal and lessened French reliance on the United States. Fears of abandonment are not unfounded: countries have honored their alliance commitments in war just 22 percent of the time since 1945.\footnote{\cite{berkemeierRP18}. See also \cite{Leedsetal2000}.} To protect other states, an alliance member must overcome skepticism about its resolve to defend an ally.\autocite[36]{schellingarms66} %As Thomas Schelling put it, ``the difference between the national homeland and everything `abroad' is the difference between threats that are inherently credible, even if unspoken, and the threats that have to be made credible.''\autocite[36]{schellingarms66}

Making alliance promises credible by demonstrating capability and will to intervene can be expensive. The budget hawk perspective emphasizes three costly measures that the United States takes to increase the credibility of its alliance commitments.  

\subsubsection*{Military Technologies and Equipment}

The United States must develop and procure military technologies to effectively defend allies. %The country is separated from any other continental power by two large oceans. %The so-called ``stopping power of water'' has helped keep America relatively safe since the country's inception.\autocite{mearsheimertragedy01} 
The large distance between the United States and its allies means that Washington needs considerable power projection capabilities. As David Ochmanek, who served as Deputy Assistant Secretary of Defense for Force Development during the Obama administration, put it, without the capacity to project power ``over intercontinental distances \ldots the credibility of the U.S. deterrent and of U.S. alliance commitments would erode.''\autocite[3]{ochmanekRAND18} The United States must maintain capabilities that allow the military to move people and equipment over large distances. Weapons systems that can hit faraway targets -- including intercontinental ballistic missiles, submarines, aircraft carriers, and long-range bombers -- are also critical for this goal. 
%https://www.rand.org/content/dam/rand/pubs/perspectives/PE200/PE260/RAND_PE260.pdf 

Power projection is expensive. The U.S. Navy's Columbia-class program, for example, which intends to build 12 ballistic missile submarines, has an estimated acquisition cost of \$103 billion.\autocite{crs2020} The first Ford Class aircraft carrier, which was delivered in May 2017, cost \$12.9 billion.\autocite{gao201706} The Air Force is currently upgrading its land-based ICBM force of around 450 Minuteman III missiles at a total cost between \$85 and \$100 billion.\autocite{trevithickTD17} 

These power projection capabilities, as well as many others, are located on U.S. territory or deployed at sea. Despite their location, the United States could use them to defend allies. In response to a hypothetical North Korean nuclear strike against Seoul, for example, the United States could level any North Korean city using ICBMs launched from Montana, North Dakota, or Wyoming. However, domestic and sea-based deployments may be insufficient for deterrence and reassurance in some situations. This brings us to our next point: to make its alliance promises more effective, Washington may need to deploy personnel and weapons on allied territory. 

%https://carnegieendowment.org/files/low_numbers.pdf

\subsubsection*{Overseas Bases}

Washington maintains a vast network of overseas military bases. %The United States began to expand its permanent military presence abroad after World War II, and it maintains a sizable number of bases on foreign soil today. 
The most recent Department of Defense (DOD) \textit{Base Structure Report} indicates that the United States has 514 bases in 45 foreign countries.\autocite[7]{BSR18} Nearly 80 percent of these bases are located in three U.S. allies: Germany, Japan, and South Korea. U.S. overseas bases vary in size. About five percent (24) of them are classified by DOD as large sites valued at more than \$2 billion.\footnote{This is based on the sites' permanent replacement value (PRV).} %For example, Ramstein Air Base in Germany, a massive complex with 742 buildings on more than 3,000 acres, is worth \$12.6 billion.\autocite[77]{BSR18}

%https://www.acq.osd.mil/eie/Downloads/BSI/Base%20Structure%20Report%20FY18.pdf

%The United States uses its overseas bases to forward-deploy troops and military equipment. At the post-World War II peak deployment period in the late-1960s, the United States had more than 1 million troops stationed abroad.\autocite{kaneglobal04} About half of these were in Vietnam to support the war effort; most of the others were stationed on the territory of an ally. In 1968, for example, the United States had about 214,000 troops in Germany and 83,000 in Japan.\autocite{kaneglobal04} Trump campaigned on reducing the U.S. footprint abroad, but recent data show that there are 194,000 troops deployed on foreign soil -- a modest two percent decline from the end of the Obama administration.\autocite{macdonaldFA19} Washington also deploys aircraft, missiles and other technology to overseas bases. It has even stationed nuclear weapons on the territory of 14 countries since 1945, and continues to store B-61 gravity bombs at bases in Belgium, Germany, Italy, the Netherlands, and Turkey.\autocite{FuhrmannSechser2014}
%https://www.heritage.org/defense/report/global-us-troop-deployment-1950-2003
%https://www.foreignaffairs.com/articles/2019-12-03/trump-didnt-shrink-us-military-commitments-abroad-he-expanded-them

Using overseas bases to forward-deploy troops and military equipment is expensive. A 2019 Congressional Budget Office analysis found that the annual base operations support (BOS) costs of overseas bases were 27 percent higher, on average, than the BOS costs for domestic sites even after controlling for base characteristics such as geographic size and the number of employees.\autocite[11]{cbo201911} Operating a single overseas base entails between \$50 million and \$200 million per year in fixed costs plus anywhere from \$10,000 to \$40,000 per person in variable costs.\autocite[xxv]{randoverseas13} The total cost of basing abroad may be as high as \$85 to \$100 billion each year.\footnote{\cite{vinebase15}. These figures are based on data for 2014.} %To budget hawks, 

Based on one line of thinking, these expenses are necessary to maintain healthy alliances.  Overseas military deployments augment the capacity of the United States to defend its allies in the event of an attack. But they may also add credibility to American security assurances. Foreign deployments can serve a ``tripwire'' function. An invasion of a U.S. ally would probably target U.S. military personnel and equipment on the ally's soil. Any loss of American life would make it exceedingly difficult for the United States to remain on the sidelines during the conflict. Placing forces on an ally's territory thus increases U.S. political will to intervene on its ally's behalf in the alliance is challenged.\autocite[47]{schellingarms66} %Schelling applied this logic to the small U.S. garrison in Berlin during the Cold War: those forces can keep the Soviet army at bay, he wrote, because ``they can die \ldots in a manner that guarantees that the action cannot stop there.''\autocite[47]{schellingarms66}

In addition, military bases provide a visible symbol of U.S. resolve to defend an ally.\autocite[7]{cooleybase08} General Vincent Brooks, the United States Forces Korea (USFK) commander, underscored this point when discussing the garrison at Pyeongtaek that is expected to house 45,000 Americans by 2022: the base is a ``significant investment in the long-term presence of U.S. Forces in Korea,'' he said, and provides ``living proof of the American commitment to the alliance.''\autocite[Quoted in][]{hincksT18} Base costs, then, may be a necessary feature of alliances. By ``sinking costs'' to defend an ally, the United States signals that it is determined to uphold its promise, since a less resolved country would not incur that expense.\autocite{Fearon1997,FuhrmannSechser2014}


\subsubsection*{Alliance Coordination}

U.S. alliances require coordination, as coalition forces must be able to fight together in order to be militarily effective.\autocite[On the challenges of military coalitions, see, for example,][]{krepscoalitions11,wolfordpolitics16} Taking measures to facilitate joint warfighting in peacetime also bolsters deterrence by signaling to third-parties that the United States and its partners are ready for a coordinated fight.\autocite[2]{poastarguing19} Yet fighting jointly is often challenging because each alliance partner may have unique military strategies and doctrines, as well as different weapons and means of communication.\autocite{lingreenbergTNSR20}
%https://tnsr.org/2020/03/allies-and-artificial-intelligence-obstacles-to-operations-and-decision-making/

To improve alliance coordination, countries sometimes establish a joint headquarters or military command. The NATO Command Structure (NCS), for example, includes a permanent multinational headquarters that harmonizes military actions across the alliance. This effort carries substantial personnel and infrastructure costs: NATO had 22,000 staff across 33 commands when the Soviet Union collapsed and currently has 6,800 personnel in seven commands.\autocite{natofactsheet201802} Joint military exercises are another way of improving alliance cohesion. NATO, for instance, conducted 102 military exercises in 2019 to ``ensure that NATO forces are trained, able to operate together and ready to respond to any threat from any direction.''\autocite{natofactsheet201902} Measures such as these enhance cohesion but also necessitate additional spending. 
%https://www.nato.int/nato_static_fl2014/assets/pdf/pdf_2018_02/1802-Factsheet-NATO-Command-Structure_en.pdf
%As NATO puts it, the NCS is the ``backbone'' of the alliance that allows all members to ``participate in, and contribute to, the command and control of all Alliance operations, missions, and activities across all military domains.''\autocite{natofactsheet201802} 

%Joint military exercises are another way of improving alliance cohesion. The United States regularly carries out regular exercises with alliance partners. NATO conducted 102 military exercises in 2019 to ``ensure that NATO forces are trained, able to operate together and ready to respond to any threat from any direction.''\autocite{natofactsheet201902} In Latin America, Washington has orchestrated the UNITAS maritime exercises with regional alliance partners every year since 1960.\autocite{socom17} Joint exercises in the Asia-Pacific region -- particularly with Australia, Japan, and South Korea -- are also a staple. Budget hawks sometimes complain about the cost of these exercises. In June 2018, President Trump cited cost savings as a rationale for canceling the Freedom Guardian military exercise with South Korea: ``We save a fortune by not doing war games,'' Trump wrote on Twitter.\autocite[Quoted in][]{reuters20180707} This particular exercise reportedly cost \$14 million -- a significant expense but not as substantial as some of the other direct costs of defense commitments.\autocite{reuters20180707} 
%https://www.nato.int/nato_static_fl2014/assets/pdf/pdf_2019_02/1902-factsheet_exercises_en.pdf
%https://www.southcom.mil/Media/Special-Coverage/UNITAS-2017/
%https://www.reuters.com/article/us-northkorea-usa-military-cost/cost-of-one-of-those-expensive-u-s-south-korea-military-exercises-14-million-idUSKBN1JW348

\subsection*{Compensating for Free-Riding}

The direct costs of protection are the most obvious but they are not the only ones, according to the budget hawk school. Allies may be able to ``free-ride'' on the United States by reducing their defense expenditures, requiring Washington to shoulder a disproportionate share of the burden of providing security. The notion of alliance free-riding dates back to work by Mancur Olson and Richard Zeckhauser.\autocite{OlsonZeckhauser1966} They argued that military alliances provide public goods, as collective defense is non-excludable, so all alliance partners benefit from security no matter how much they contribute. As a result, U.S. allies can safely reduce their defense spending as long as Washington protects them. %\footnote{Public goods are also characterized by non-rival consumption, which means that one person's ability to ``consume'' the good does not decline as more people benefit from it. Clean air, for example, remains pure no matter how many people breathe it.} 
%In this way, being safe from international threats is like breathing clean air or watching public broadcasting services. As a result, U.S. allies can safely reduce their defense spending as long as Washington is willing to provide protection. Free-riding is thought to be particularly prevalent in asymmetric alliances where a major power like the United States promises to defend smaller states. 

U.S. officials from both major political parties have long complained about free-riding.  Barrack Obama, for example, put it bluntly in a 2016 interview: ``free riders aggravate me.''\autocite[Quoted in][]{goldbergA16} These complaints have some justification, as U.S. allies do under-invest in defense, at least under some circumstances.\autocite[See, for example,][]{PluemperNeumayer2015, Fuhrmann2020} Free-riding is a problem for the United States, based on the budget hawk school of thought, because Washington must spend more to compensate for disproportionately small allied contributions.\footnote{Diehl, ``Substitutes or Complements?,'' 167; and Posen, \textit{Restraint}, 34.} U.S. leaders may try to overcome free-riding by encouraging allies to share more of the security burden, but even when these inducements work they can require additional expenditures. For example, the United States sometimes uses side-payments to persuade allies to join a military coalition.\autocite[3]{wolfordpolitics16} %Consider U.S. efforts to coax Turkey into allowing troops on its soil in support of the 2003 Iraq War: Washington offered \$6 billion in aid and billions more in loans.\autocite{mcclureS03}
%https://www.salon.com/2003/03/12/foreign_aid/
%\autocite[167]{Diehl1994},[34]{Posen2014}  
%https://history.state.gov/historicaldocuments/frus1961-63v13/d168
%https://warontherocks.com/2017/08/a-history-of-vexation-trumps-bashing-of-nato-is-nothing-new/
%https://www.theatlantic.com/magazine/archive/2016/04/the-obama-doctrine/471525/
%https://www.whitehouse.gov/briefings-statements/remarks-president-trump-cabinet-meeting-12/
%In January 1963, John F. Kennedy privately told his advisers, ``we cannot continue to pay for the military protection of Europe while the NATO states are not paying their fair share and living off the `fat of the land.'''\footnote{\autocite{fruskennedy}. See also \autocite{creswellWOTR17}.}
%Trump has made particularly forceful statements along these lines. ``You can call them allies if you want,'' he said in a cabinet meeting in January 2019, ``but a lot of our allies were taking advantage of our taxpayers and our country.  We can't let that happen.''\autocite{trumpremarks190102} 

%Posen encapsulates this sentiment: ``The limited efforts of its principal allies have several negative consequences for the United States. The most obvious negative consequence is that the United States overspends on its military. If these countries did more, the United States could do less.''\autocite[34]{Posen2014} 
%As Paul Diehl put it, ``the major power may have to increase military spending in order to protect its weak ally against the other major power rival; it is not likely that the minor power will assume most (or even its proportional share) of that burden.''

% JA: I think this whole paragraph can go- brings things closer to the big debate than we want
%The problem of free-riding is a major reason scholars from the restraint camp think that America should ``come home.'' Eugene Gholz, Daryl Press, and Harvey Sapolsky advanced this idea more than 20 years ago:  ``The allies, now in the same economic league as America, should discover the full cost of their defense while the United States turns to long-avoided problems with its infrastructure, education system, budget deficit, and race relations.''\autocite[17]{gholzIS97}

%Diehl, 183: “complementary effects may be manifest when alliance membership entails additional mili- tary requirements on the allies and when a major power joins an asymmetric alliance with free-riding by the smaller partners.”

\subsection*{Bailing Out Reckless Allies}

A related concern is that U.S. allies may pull Washington into unwanted military disputes. Knowing that they have U.S. protection, weaker allies may take more risks and provoke military confrontations. %Brett Benson shows that unconditional pledges of military support increase the likelihood that revisionist states will challenge the status quo militarily.\autocite{Benson2012} 
For example, Chiang Kai-shek felt emboldened by U.S. defense promises during the two serious Taiwan Strait crises of the 1950s.\autocite{Benson2012} The moral hazard problem in military alliances has implications for the U.S. defense burden. If allied provocations lead to conflict with a third-party, Washington might be forced to launch a costly intervention to bail out its partner or defuse the conflict.\autocite[See][44]{Posen2014} In the absence of credible security guarantees, the United States would not have to shoulder these costs.

%This problem is known as ``moral hazard'' and it applies to many contexts in everyday life. Wearing seatbelts can keep people safe, but it can also increase the risk of a tragic accident by making a driver feel invulnerable. So it is with military alliances: they can provoke conflict, in addition to deterring it. 

%Benefits and costs of bases: https://www.rand.org/content/dam/rand/pubs/research_reports/RR200/RR201/RAND_RR201.sum.pdf
%``We found that there are annual recurring fixed costs to having a base open, rang- ing from an estimated $50 million to about $200 million per year, depending on ser- vice and region, with additional variable recurring costs depending on base size. ''

%another base/troop cost estimate: https://www.politico.com/magazine/story/2015/06/us-military-bases-around-the-world-119321
%troop data https://www.pewresearch.org/fact-tank/2017/08/22/u-s-active-duty-military-presence-overseas-is-at-its-smallest-in-decades/

%As Schelling put it during the cold war, ``It hardly seems necessary to tell the Russians that we should fight them if they attack \textit{us}. But we go to great lengths to tell the Russians that they will have America to contend with if they or their satellites attack countries associated with us.''\footnote{\citet[35]{schellingarms66} 

\section*{Bargain Hunters: U.S. Alliances Save Money}

The bargain hunter perspective offers an alternative view: that U.S. alliance commitments generate costs savings or modest increases in U.S. defense spending.\autocite[See, for example,][78]{rapphoopershields20} As with the budget hawk school, we strive to put forth the strongest possible argument in defense of this perspective. Not every scholar or policymaker in this camp advances all of the arguments we include in this school, but they all support the same overarching conclusion. 
%The bargain hunter line of thinking rests on three core claims: alliances divide labor and generate efficiencies; defense pacts save money by reducing conflict; and the United States can maintain alliance credibility without massive military investments. Not every scholar or policymaker in this camp advances all of these arguments, but they all support the same overarching conclusion. 
%power projection is not all about protecting allies

%Rapp-Hooper exemplifies this view: ``U.S. alliance strategy was not nearly so costly an endeavor as we might expect \ldots If the United States had not extended its Cold War alliances, or if it had ended them in an effort to reduce political and material costs, it may not have saved much.''\autocite[78]{rapphoopershields20} Elbridge Colby and Jim Thomas similarly warn that ``shedding alliance commitments could result in the United States having to pay higher military `insurance premiums.'''\autocite[37]{colbyTNI16}



%MF: I changed the order in which we present the mechanisms
\subsection*{Alliances Generate Efficiencies}

The budget hawk perspective emphasizes how weaker alliance partners free-ride on the United States. By contrast, bargain hunters claim that the free-riding logic is flawed. Free-riding is possible if alliances generate public goods. Scholars have long recognized, however, that the benefits of alliances might be excludable.\autocite[See, for example,][]{Sandler1993} It is prohibitively expensive to allow some people to breathe clean air while preventing others from doing so. By contrast, U.S. investments in extended deterrence may benefit some allies but not others. The substantial military presence in Germany, for example, is probably insufficient to stop a Russian invasion of Estonia. Moreover, Washington could decide to abandon an ally. Viewed from this perspective, free-riding is a risky strategy.\autocite{Fuhrmann2020} 

There is evidence consistent with this view: free-riding is generally rare\autocite{Alley2021} and free-riding on the United States is less common than the budget hawk perspective suggests.  There is likely within-country variation in free-riding among NATO allies. Some leaders -- particularly those with high self-efficacy and feelings of power -- free-ride on the United States by cutting defense expenditures, but others make sizable contributions to collective defense and emphasize paying their fair share.\autocite{Fuhrmann2020}

Bargain hunters also contend that alliances reduce \textit{U.S.} military spending by generating effeciencies. Even as allies rely on the United States, Washington gets valuable help from its partners. As Brooks, Wohlforth, and Ikenberry argue, we should think about alliances through a bargaining framework -- not a public goods one -- whereby allies divide labor as ``part of a complex hegemonic bargain.''\autocite[28]{Brooksetal2013} In this view, allies often divide labor in ways that benefit the U.S. bottom line, compared to a situation where Washington had no partners. Obama acknowledged this in the same 2016 interview where he complained about free-riding, saying: ``We don't have to always be the ones who are up front \ldots Sometimes we're going to get what we want precisely because we are sharing in the agenda.''\autocite{goldbergA16}
%https://www.theatlantic.com/magazine/archive/2016/04/the-obama-doctrine/471525/

%Unilateral action requires the United States to bear 100 percent of the burden. 
Alliance partners often assist in U.S.-led military operations, and their contributions reduce U.S. costs.\autocite{krepscoalitions11} To illustrate, during Operation Enduring Freedom, the post-9/11 campaign against the Afghan Taliban, more than 3,000 foreign troops joined 19,000 U.S. forces.\autocite{DOS20060131} Aside from troops, the White House acknowledged substantial contributions from allies in support of combat operations: France provided 24 percent of its total naval forces, including its only carrier battle group; Italy also provided a carrier battle group and 13 percent of its total naval capabilities; Great Britain gave the coalition its only Tomahawk Land Attack Missiles (TLAMs); and Turkey provided KC-135 refueling support for U.S. planes in transit.\autocite{WHnd} 
%In some cases, allies make small symbolic contributions. Yet even if allies do just 10 percent of the work, the United States is better off than if it had no help. Working with allies imposes coordination costs, but allied efforts might offset these costs and generate net savings for the United States. 

Furthermore, some allies directly offset the cost of U.S. bases on their territory. South Korea, Germany and Japan all provide the United States with cash contributions, reduced rent, and waivers of taxes and damage fees. Identifying the precise magnitude of these contributions is difficult, but one estimate suggests that Japan offsets between \$2 and \$6 billion of the annual cost of U.S. bases.\autocite[143-49]{randoverseas13} Along similar lines, allies sometimes make direct financial contributions to support U.S. military endeavors. Although Germany did not provide troops during the Persian Gulf War, it gave \$1 billion to the U.S.-led war effort in 1990 and promised an additional \$5.5 billion to the United States in 1991.\autocite{goshkoWP91}
%https://www.washingtonpost.com/archive/politics/1991/03/27/germany-to-complete-contribution-toward-gulf-war-costs-thursday/8af9f5b5-ef7d-4b0c-84cb-2afcdbf522d6/

Allies also provide unique capabilities, resources, or expertise that the United States would otherwise have to build itself. Estonia, for example, does not offer much conventional military capability, but it has specialized knowledge in cyber warfare, which has helped formulate cyber policy within NATO.\autocite{lingreenbergTNSR20} U.S. allies also have superior minesweeping capabilities. Clearing mines was critical during the Persian Gulf War in order to maintain safe shipping routes. U.S. allies led the way in this task: they cleared more than 1,000 mines in the Gulf, while the U.S. Navy destroyed just 248.\autocite{thompson2013lessons} Though larger U.S. investments in minesweepers are possible, relying on allies makes them unnecessary. 
%https://tnsr.org/2020/03/allies-and-artificial-intelligence-obstacles-to-operations-and-decision-making/

%https://books.google.com/books?id=Hr-uaYXoyIQC&pg=PT37&lpg=PT37&dq=allies+provide+equipment+minesweeping+gulf+better+than+the+united+states&source=bl&ots=rckzc2pbgU&sig=ACfU3U3TZOxk775HW-FhJjH9aacmeokECw&hl=en&sa=X&ved=2ahUKEwj1mI-vgo7qAhVP-qwKHQ9FApMQ6AEwC3oECAoQAQ#v=onepage&q=allies%20provide%20equipment%20minesweeping%20gulf%20better%20than%20the%20united%20states&f=false

% JA: cut this because foreign aid doesn't come out of the defense budget. Not sure about peacekeeping, but I think that's direct through UN contributions, which is also outside DoD. 
%In addition, U.S. allies spend more than Washington on peacekeeping and foreign aid. The United States spends a mere 0.3 percent of its GDP on these kinds of international affairs programs.\autocite[153]{beckleyunrivaled18} By contrast, many U.S. allies spend more than 0.7 percent of their GDP -- the benchmark set by the United Nations -- on foreign aid alone. This includes Denmark, the Netherlands, Norway, Sweden, and the United Kingdom.\autocite{mcbrideCFR18} When looking at these contexts, the United States -- not its allies -- looks like the ``free-rider.''\autocite[28]{Brooksetal2013}

\subsection*{Reducing Conflict Lessens the U.S. Budgetary Burden}

Budget hawks fear that alliances will pull the United States into conflicts, necessitating further military expenditures. However, alliances may instead restrain partners. U.S. allies value protection from a nuclear-armed superpower and understand that acting recklessly may lead to alliance termination or rollback. U.S. allies therefore have strong incentives avoid picking fights or adopting aggressive foreign policies. Consistent with this view, some studies show that defense pacts with nuclear powers like the United States do not increase -- and may decrease -- the likelihood that a state will instigate military disputes.\autocite{narangJCR19, fuhrmannnuclear14}
%The Cuban missile crisis offers a cautionary tale for allies seeking nuclear protection. After Fidel Castro called for nuclear war with the United States, Soviet leader Nikita Khrushchev became convinced that his capricious counterpart was untrustworthy and Moscow never extended a formal defense pact to Cuba. 

Alliances also promote peace through extended deterrence, especially if they include a nuclear power.\autocite{Leeds2003,FuhrmannSechser2014} Making an investment in an alliance, therefore, can save money by avoiding wars or crises that might otherwise happen.\footnote{\cite[18]{BrandsFeaver2017}. See also \cite[37]{colbyTNI16}.} Saddam Hussein invaded Kuwait in 1990, in part because he doubted that the United States would fight to reverse Iraq's territorial gains. Would Saddam have reached a similar conclusion if the United States had a formal defense pact with Kuwait? If so, such an alliance would have prevented the subsequent loss of blood and treasure in the Persian Gulf War. To the bargain hunter school, prevention is cheaper than treatment. 
%As Brands and Feaver write, ``if the United States pulled back from its alliance commitments and waited for a crisis to develop before surging back into key regions, it might find such a mission more difficult -- and more expensive -- than simply protecting its allies in the first place.'

\subsection*{Some Direct Costs Are Unnecessary for Alliance Credibility}

The budget hawk perspective claims that alliance credibility depends on sinking tremendous costs. In this view, a defense pact will not deter aggression unless the United States spends billions of dollars on military equipment for allies, overseas bases, and preparations for joint warfighting. Demonstrating resolve to defend an ally is undoubtedly important for extended deterrence and reassurance -- but sinking costs is not the only way to establish credibility.
%Doing so augments the overall capabilities of the alliance and signals to third-parties that Washington is determined to defend its friends. 

Hand-tying signals can also convey information about a country's willingness to fight on behalf of an ally.\autocite[On the difference between hand-tying and sunk costs, see][]{Fearon1997} Unlike sinking costs, tying hands is not immediately costly but can become expensive if a commitment is challenged. Public threats are a classic hand-tying signal.\autocite[See, for example,][]{schultzdemocracy01}
Signing an alliance treaty can also be an effective hand-tying device. Although there are some up-front diplomatic and political costs associated with negotiating an alliance treaty, the biggest costs come into play if an ally is attacked. A state could, of course, abandon their ally and avoid a costly war. But countries that shirk their alliance promises damage their reputation.\autocite{giblerJCR08,crescenziISQ12} Potential attackers know that countries want to protect their reputations, which contributes to the credibility of a public defense pact.

%Unlike sinking costs, tying hands is not immediately costly but can become expensive if certain contingencies arise. Public threats are a classic hand-tying signal.\autocite[See, for example,][]{fearonAPSR94,schultzdemocracy01} Making a public threat is not costly in the moment, but a leader who backs down could pay domestic or international audience costs. %\footnote{The existence and magnitude of audience costs is still debated in scholarship. For critical perspectives, see, for example, \autocite{downesIO12,snyderAPSR11}.} 
%Potential challengers take public threats seriously, based on the logic of audience costs, because they understand that a less resolved leader would not make them.%\footnote{This is especially true for democracies, although leaders in certain non-democratic states can also generate audience costs. See \cite{weeksIO08}}

%Signing an alliance treaty can also be an effective hand-tying device. Although there are some up-front diplomatic and political costs associated with negotiating an alliance treaty, inking such a deal has few costs. The real cost comes into play only if an ally is attacked. A state could, of course, abandon their ally and avoid a costly war. But countries that shirk their alliance promises damage their reputation. Indeed, treaty violation may make it harder to find alliance partners in the future.\autocite{giblerJCR08,crescenziISQ12,naranginternational12} Many analysts and policymakers see a state's reputation as one of its greatest assets. Schelling put it in particularly stark terms: ``We lost thirty thousand dead in Korea to save face for the United States and the United Nations \ldots and it was undoubtedly worth it.''\autocite[124]{schellingarms66} Potential attackers know that countries want to protect their reputations, which contributes to the credibility of a public defense pact.

There is evidence that hand-tying is effective in an alliance context -- perhaps more so than sinking costs. Matthew Fuhrmann and Todd Sechser find that countries that have defense pacts with nuclear powers are less vulnerable to military aggression than their counterparts that lack this protection.\autocite{FuhrmannSechser2014} However, conditional on the existence of a defense pact, their analysis shows that sinking costs by deploying nuclear weapons on the ally's territory does not significantly add to extended deterrence. U.S. allies also seem to value the alliance promise itself more than military deployments. Similarly, Jiyoung Ko's experimental analysis shows that the South Korean public feels reassured by U.S. public declarations of protection but that forward-deployment of nuclear forces do not bolster this effect.\autocite{koFPA18} This evidence suggests that the United States can achieve credibility without expensive military deployments overseas -- as long as it maintains the capabilities needed to fight on the ally's behalf and states its intentions publicly. The overall cost of U.S. alliances, then, may be smaller than it initially appears.

%It is therefore not surprising that we see heterogeneity in cost-sinking across U.S. alliances. All else equal, alliances will be cheaper when the United States favors hand-tying over sinking costs as a means to enhance credibility. 
Many analysts assume that all U.S. defense commitments are NATO-like arrangements with high direct costs. But many U.S. commitments are more limited. Washington is obligated by treaty to defend most of Latin America from external attack, but there is not a huge U.S. military presence on the territory of those states. In 2000, for example, the United States had just over 2,000 troops stationed in all of South America and the Caribbean and 112,000 in Europe.\autocite{kaneglobal04}  When making the case that alliances are expensive, budget hawks highlight the priciest commitments. As some alliance commitments have modest direct costs, alliances may not generate a large financial burden on average. % All else equal, alliances will be cheaper when the United States favors hand-tying over sinking costs to enhance credibility. 
%This comparison underscores an important reality: it is not alliances per se that generate high direct costs but rather what the United States does with them once they are formed.\autocite[84-85]{rapphoopershields20}

On top of this, much of the cost-sinking discussed by budget hawks is not causally connected to military alliances, based on the bargain hunter line of thinking. The United States projects power in order to deter attacks against its homeland, broadly influence the behavior of its adversaries, and ensure open access the the global commons -- not just protect allies. Expenses that are seemingly made on behalf of allies also provide private benefits to the United States. Lt. Gen. Frederick Hodges, who once served as the top U.S. Army commander in Europe, put it bluntly: ``The reason we have troops overseas in Germany is not to protect Germans, everything we have is for our benefit.''\autocite[Quoted in][]{crowleyNYT20} 
%As one indication of this, U.S. drone strikes in Pakistan, Yemen, and Somalia depend on Ramstein Air Base.\autocite{frankeCNAS16} 

% Start comment on power projection section
\begin{comment}
% JA: this isn't quite about spending- more about what alliances are for, and is more a part of the grand strategy debate. Could perhaps cut it. That or it would help to draw out the implication that the U.S. would spend just as much without alliances in the last paragraph. 

%MF: I see what you mean. I think we need to make this point but it doesn't need a whole subsection. I folded the key point into a single paragraph and put it above in the section on sinking costs. 

\subsection*{Power Projection Is Not All About Allies}

Alliances are not solely responsible for the costs of power projection. The United States projects power in order to deter attacks against its homeland, broadly influence the behavior of its adversaries, and ensure open access the the global commons -- not just protect allies. Expenses that are seemingly made on behalf of allies also provide private benefits to the United States. This is especially true of overseas bases. 

If bases were intended only to defend the host country, they should be located exclusively in states that Washington has pledged to defend by formal treaty. Yet Washington operates bases in countries with whom it does not share formal defense pacts, such as Djibouti, Kuwait, and Qatar. These bases primarily support \textit{American} security objectives -- not the host country's. Al Udeid Air Base in Qatar, for example, serves as a staging area for U.S. military operations in Afghanistan, Iraq, Syria, and Yemen.\autocite{taylorWP19} %In addition, the high-profile U.S. operation that killed Iranian Gen. Qassem Soleimani in January 2020 was reportedly run out of al Udeid.\autocite{dilanianNBC20}  
%https://www.washingtonpost.com/world/as-trump-tries-to-end-endless-wars-americas-biggest-mideast-base-is-getting-bigger/2019/08/20/47ac5854-bab4-11e9-8e83-4e6687e99814_story.html}
%https://www.nbcnews.com/news/mideast/airport-informants-overhead-drones-how-u-s-killed-soleimani-n1113726

Similarly, bases on the territory of allies serve U.S.-specific interests, in addition to promoting extended deterrence. Consider the U.S. military presence in Germany. Lt. Gen. Frederick Hodges, who once served as the top U.S. Army commander in Europe, put it bluntly: ``The reason we have troops overseas in Germany is not to protect Germans, everything we have is for our benefit.''\autocite[Quoted in][]{crowleyNYT20} As one indication of this, U.S. drone strikes in Pakistan, Yemen, and Somalia depend on Ramstein Air Base.\autocite{frankeCNAS16} 

The capabilities that Washington needs may rise along with the number of countries under its defense umbrella. Some have argued, for example, that the United States needs a larger nuclear arsenal in order to credibly protect its allies.\autocite[See the discussion in Chapter 2 of][]{actonlow11} However, based on this perspective, at least some spending that seems to protect allies in fact satisfies other U.S. national security objectives. 
%https://carnegieendowment.org/files/low_numbers.pdf

\end{comment}

\section*{Testable Predictions}


The preceding discussion offers two contrasting views about the financial costs of U.S. alliance commitments. How do we know which view is correct? The first step in answering this question is to identify testable hypotheses that follow from each view. 
%The budget hawk perspective suggests that alliances drastically increase U.S. defense expenditures, while the bargain hunter view holds that defense pacts are relatively cheap and may actually save money. 

\singlespacing

\noindent \textbf{Budget Hawk Hypothesis}: Alliance commitments increase U.S. defense expenditures.

\vspace{1em}

\noindent \textbf{Bargain Hunter Hypothesis}: Alliance commitments do not reliably increase, and may decrease, U.S. defense expenditures.

\doublespacing

A negative or null relationship between alliance commitments and defense spending would clearly confirm the bargain hunter perspective, while contradicting the budget hawk view. However, a positive relationship would not necessarily refute the bargain hunter perspective. Some bargain hunters argue that alliances have a net-neutral effect on the defense budget as long-term savings offset short-run costs.\autocite[See, for example,][]{rapphoopershields20,colbyTNI16} Others accept that alliances may raise U.S. defense spending but contend that these increases are relatively small. Therefore, we consider the direction and magnitude of the relationship between alliances and military spending. 

It is also possible that both perspectives are correct to some degree. U.S. alliance commitments may increase spending in some ways and generate cost savings in others. We assess the net effect. Furthermore, a substantively large effect would not necessarily mean that the bargain hunter arguments are wrong -- just that the financial burdens of U.S. alliance commitments swamp any cost savings.


\section*{Research Strategy: Getting to the Truth}

Our research objective is to identify both the direction and the magnitude of the relationship between alliance commitments and U.S. defense spending. This is a challenging task because the U.S. defense budget does not clearly delineate alliance-specific costs. Some costs that are seemingly straightforward to calculate, such as the personnel and equipment costs associated with defending Europe, are not clearly specified in the budget.\autocite{posenTNI16} On top of this, as the preceding discussion underscores, many of the financial costs and benefits of alliance commitments could not plausibly have a budgetary line-item. %There is no way to isolate the price the United States pays due to allied free-riding in the DOD budget, for instance. 

One solution is to calculate the total cost of U.S. grand strategy rather than the burden of alliances per se. Taking this approach, Posen estimates the cost of alliances based on the savings the United States could generate if it abandoned the force structure necessary to maintain them.\autocite{Posen2014} However, we want to separate the costs of commitments from the broader financial burden of U.S. grand strategy, which has been relatively constant after 1945. The United States seeks global power projection capabilities to defend many of its own interests, not just defend allies. As a result, looking at the total force structure costs could overstate the financial toll of defense pacts. On the other hand, this could also miss some of the ``hidden'' costs associated with free-riding and moral hazard, thereby understating the true financial burden. We therefore need a different approach. 

To estimate the financial costs of U.S. alliances, we rely on temporal variation in the number of states with a formal U.S. security guarantee. After making zero formal defense commitments between 1779 and World War II, the United States entered a new era and began offering treaty-backed protection to other countries. This started with the Act of Chapultepec in 1945 (which led to the Rio Pact two years later) and continues through today as NATO expands. Although the number of states under U.S. protection has generally increased over the last 70 years, there have been decreases too. The United States scrapped its defense commitment to Iran, for example, after the 1979 Islamic Revolution. In light of this variation, we can estimate how changes in the number of defense commitments alter U.S. defense spending over time. 

We do so with a statistical model of U.S. military spending from 1947 to 2019. The model allows us to estimate the impact of alliances on budgetary changes over time by adjusting for confounding variables, such as the president's political party. This approach will pick up the full financial costs and benefits of U.S. alliances, including those that are difficult to directly observe.%\footnote{With such observational data, we cannot make clear causal claims about the relationship between alliance commitments and military spending.}





