
%``Moreover, even under the rosiest scenarios, curtailing America’s alliance military com- mitments would save something in the vicin- ity of 1 percent of gdp''
%https://www.jstor.org/stable/26557320?casa_token=ej5vXAXtQx4AAAAA%3Ak3OHJPMegfFA6ZFOZwEVMasbLm_RSDLUU5dukz1LNhLRmyUOZKOMhA0GUyEk783AZtYmsdRQWFbuK3J85j9Qqua6xsWSl3LE0bWzZQv9cqTV5N4Bcw&seq=1#metadata_info_tab_contents

%If the United States could achieve its military policy goals, other than pro-moting global stability, with a $150 billion defense budget—that is, if the United States only requires the rest of the force to maintain the alliances and sustain the operations that enhance stability—then the budgetary cost of the stability mission is $150 billion.
%https://www.tandfonline.com/doi/pdf/10.1080/09636410108429444?casa_token=sfy6T7MDJrAAAAAA:d0hHmbbUY7Gol5zZJq55E9kwPVkcPPLC3UYtI1jzeNxDDIrrRsXjtwitRDGXGNqCcW88tWsfUoE , p. 54
 
%Posen: ``I have argued that if the United States were more judicious in its promises abroad, perhaps a fifth of the defense budget could be cut (excluding the costs of actual wars), amounting to roughly one hundred billion dollars per year at current prices. ''

%Some figures on the costs of overseas troop deployments: https://theconversation.com/why-does-the-us-pay-so-much-for-the-defense-of-its-allies-5-questions-answered-127683

%From Beckley: ``14. Eugene Gholz, Daryl G. Press, and Harvey M. Sapolsky, “Come Home, America: The Strategy of Restraint in the Face of Temptation,” International Security, Vol. 21, No. 4 (Spring 1997), pp. 5–48; Layne, The Peace of Illusions, chap. 8; Ted Galen Carpenter, Smart Power: Toward a Prudent Foreign Policy for America (Washington, D.C.: CATO Institute, 2008); and Benjamin H. Friedman, Brendan Rittenhouse Green, and Justin Logan, “Correspondence: Debating American Engagement—The Future of U.S. Grand Strategy,” International Security, Vol. 38, No. 2 (Fall 2013), pp. 183–192.''


\section*{Discussion and Conclusion}
 
Military alliances figure prominently in debates about U.S. grand strategy and foreign policy. This article investigated the financial cost of U.S. alliance commitments. We summarized two schools of thought that offer contradictory answers: the budget hawk school and the bargain hunter perspective. The former suggests that alliances generate large increases in defense expenditures, while the latter claims that security guarantees are relatively cheap and may even save the United States money in the long run. To assess these competing predictions, we designed a statistical model to estimate how changes in the number of alliance commitments over time affect U.S. defense expenditures while controlling for confounding factors, such as the president's political party and economic growth. Our approach accounts for the financial costs and benefits of U.S. security guarantees, including things that are difficult to measure directly, such as savings from extended deterrence and spending to compensate for ``free-riding'' by allies. 

We found that increasing the number of U.S. alliance commitments is associated with greater defense expenditures. In the long-run, on average, one additional alliance commitment adds between \$11 and \$21 billion to the level of the defense budget. Our models identify the general trend -- an important first step -- but cannot tell us the true budgetary burden of any single alliance. Some alliance commitments may exceed the \$11 and \$21 billion range, while others are almost certainly below it. Our analysis showed, for example, that post-Cold War NATO expansion into the Baltic States added between \$16 and \$45 billion to the defense budget, an average of between \$5.33 and \$15 billion per country. Other U.S. alliances, such as the one with Haiti, are undoubtedly even cheaper. 

A series of follow-on analyses showed that changing our research design in various ways can increase or decrease the range of our initial estimate. In each case, however, the results reaffirmed that increases in alliance commitments are correlated with large long-run growth in the defense budget. The increase in defense spending resulting from an additional alliance commitment is smallest (\$6 to \$15 billion) when we include two lags of defense spending in our model rather than one. 

The United States obviously does not form alliances at random. It is therefore possible that the factors giving rise to security guarantees -- rather than the commitments per se -- account for our results. We addressed this by controlling for factors that affect defense spending and alliance formation and are possible to measure, like expenditures by major power rivals and presidential partisanship. To account for unobservable confounders that do not change over time within the same administration, we estimated models with presidential fixed effects. This further reduces the risk of omitted variable bias, but these results could still be off the mark if there are one or more unaccounted for factors that change over time within the same administration and also cause both alliance formation and military spending. Although we cannot be certain that our findings reflect a causal relationship, at the very least, there is a strong positive association between alliance commitments and U.S. defense expenditures.

The fixed effects analysis is helpful for evaluating a key claim made by some scholars in the bargain hunter school: that U.S. defense spending on power projection capabilities and overseas deployments reflects grand strategy rather than alliance commitments specifically. Rapp-Hooper argues, for example, that ``America's spending reflects the country's far-flung global strategy, not its alliance commitments per se.''\autocite[100]{rapphoopershields20} If that were true, we would not expect to see differences in defense spending following changes in alliance commitments \textit{within the same administration}. That is because grand strategy, as well as beliefs about military spending and foreign policy more generally, are relatively stable within administrations. However, we find that the same presidential administration increases spending after extending security guarantees to additional countries. This suggests that alliances themselves -- not just U.S. grand strategy -- are responsible for at least some U.S. defense spending for overseas endeavors.  

%We also used presidential fixed effects to account for administration-specific confounders that do not change over time. Within the same administration, when the president's beliefs about military spending and foreign policy are relatively stable, we find that taking on an additional alliance commitment is associated with a large spending increase. Our fixed effects analysis further reduces the risk of omitted variable bias, but these results could still be off the mark if there are one or more unaccounted for factors that change over time within the same administration and also cause both alliance formation and military spending. Although we cannot be certain that our findings reflect a causal relationship, at the very least, there is a strong positive association between alliance commitments and U.S. defense expenditures.
%These results may not reflect a causal relationship. 
%Scholars have argued that U.S. defense spending on power projection capabilities and overseas deployments reflects grand strategy rather than alliance commitments specifically.  Rapp-Hooper argues, for example, that ``America's spending reflects the country's far-flung global strategy, not its alliance commitments per se.''\autocite[100]{rapphoopershields20} Overseas deployments 

Overall, our findings are consistent with the budget hawk claim that alliance commitments require the United States to shoulder a significant financial burden. They contradict the bargain hunter argument that the savings generated by alliances offset whatever expenses they require, leading to a net-neutral effect on the defense budget. Alliances may result in some efficiency gains, but their financial toll swamps any such savings for the United States. 

%Bargain hunters often argue that the savings generated by alliances offset whatever expenses they require, leading to a net-neutral effect on the defense budget.\footcite[See, for example,][]{rapphoopershields20,colbyTNI16} We have shown that this is not the case. Alliances may result in some efficiency gains, but their financial toll swamps any savings that the United States derives. 

The size of the budgetary burden stemming from alliances reflects -- and may exceed -- the expectations of the budget hawk school. Posen's estimate indicates that the United States could save about \$100 billion annually by being ``more judicious in its promises abroad.''\footcite{posenTNI16} Based on our analysis, U.S. alliance commitments may be responsible for a greater share of the defense budget than this estimate implies. We find that just one additional alliance commitment is associated with an annual increase to the defense budget between one-tenth and one-fifth of the \$100 billion estimate.
%Colby and Thomas, who are more sanguine about the costs of alliances, call this estimate the ``rosiest'' scenario, meaning that this is the most the United States could possibly save by paring back its alliance commitments.\footnote{\autocite[37]{colbyTNI16}. They write specifically that curtailing alliance commitments would save 1 percent of GDP. We assume this refers to Posen's claim, which serves as the basis for the \$100 billion estimate, that the United States could reduce defense expenditures from 3.5 to 2.5 percent of GDP.}
%MF: good point. But I think we need to compare our finding to what Posen concluded. I have revised the language here.
% JA: cut this- I'm not sure that's the right extrapolation from the model- we never see even -5 alliances in a year. 
%However, based on our analysis, U.S. alliance commitments may be responsible for a greater share of the defense budget than this estimate implies. Washington could get to \$100 billion in savings by eliminating between five and 10 alliance commitments. This translates to 7.5-15 percent of the security guarantees the United States is currently obligated to fulfill. 

%Making deeper cuts to the U.S. alliance portfolio could generate savings that goes well beyond \$100 billion annually. JA: This is the extrapolation problem- want to be careful here. 



Our findings stand in stark contrast to other research on military alliances and defense spending. Scholars usually assess this relationship by analyzing the behavior of a large number of countries over time. Many prior studies that take this approach conclude that alliances \textit{reduce} expenditures.\autocite[For a study that reaches a different conclusion, see][]{MorganPalmer2003} Matthew Digiuseppe and Paul Poast, for example, find that forming at least one alliance with a democracy decreases military expenditures by as much as 17.6 percent in the short-run, on average, compared to having zero alliances with democratic states.\autocite{DigiuseppePoast2016} Our analysis shows that the United States deviates from this general trend. Unlike other countries, the United States spends billions of dollars more on defense with each additional alliance commitment it makes. That other countries reduce their expenditures after forming alliances may be partially why the United States needs to increase its own. 

With this more complete estimate of the financial cost of security commitments for the United States, analysts and policymakers can better assess the degree to which military alliances serve U.S. interests. Our analysis supports a central pillar of restraint in U.S. foreign policy while weakening the claim made by the deep engagement school that alliances are relatively inexpensive. 

This does not necessarily imply that forming and maintaining alliances is a bad idea, however. We can neither refute nor confirm restrainers' claim that alliances are \textit{too} expensive. To know whether U.S. investments in military alliances are worthwhile, we need to incorporate the benefits the United States obtains from being part of them and weigh those benefits against these and other costs. 
 
Our study is not designed to evaluate the benefits of alliance commitments for the United States. However, prior research suggests that providing security guarantees enhances U.S. interests in several ways. Alliances make the U.S. military more effective by facilitating power projection and augmenting the capabilities it can bring to bear during crises and military conflicts.\autocite[22-25]{BrandsFeaver2017} Formal security guarantees also promote global international peace and stability by enhancing extended deterrence.\autocite{leedsAJPS03,JohnsonLeeds2011,FuhrmannSechser2014} In addition, alliances help limit the international spread of nuclear weapons, which is a major threat to U.S. national security.\autocite{bleekJCR14,reiterFPA14} Having alliance partners may also enhance U.S. political and diplomatic influence, while simultaneously weakening adversaries' sway.\autocite{Morrow1991} On the economic front, alliances facilitate trade and attract foreign direct investment.\autocite{Gowa:1993aa,liJIBS10,rapphoopershields20}
%U.S. alliances are a powerful instrument of nonproliferation because they reduce allied states' need for an independent nuclear arsenal.
%\autocite[84-85]{rapphoopershields20} 

The next step in this research program is to evaluate whether the benefits are sufficient to justify an average price tag of \$11-21 billion per alliance. The deep engagement camp often claims that the benefits of U.S. alliances are large and the costs are small. This makes the ultimate conclusion clear: alliances provide a net benefit to the United States. However, because our study indicates that alliances are more expensive than the deep engagement camp acknowledges, it is less obvious that the return on investment is unambiguously positive. Given the many benefits that result from alliances, there very well might be a net positive effect for the United States. Increased trade and investment, combined with the political and diplomatic benefits of alliances, could offset the burden on the U.S. defense budget. In order to have a clearer answer, however, we need further net assessments that account for the higher budgetary burden imposed by alliances.

%It would be harder to argue that the U.S. alliance network is desirable if the cost Washington pays to maintain it exceeds the value of the benefits, even if there are some clear positive effects produced by its security umbrella. By contrast, evidence that the benefits exceed this price tag would strengthen the argument for maintaining alliances as a centerpiece of U.S. grand strategy. 

%This evidence supports the deep engagement school's conclusion that alliances benefit the United States politically and economically. At the same time, alliances are more expensive than the deep engagement camp acknowledges. The next step in this research program is to evaluate whether the benefits are sufficient to justify a price tag of \$11-21 billion per alliance. It would be harder to argue that the U.S. alliance network is desirable if the cost Washington pays to maintain it exceeds the value of the benefits, even if there are some clear positive effects produced by its security umbrella. By contrast, evidence that the benefits exceed this price tag would strengthen the argument for maintaining alliances as a centerpiece of U.S. grand strategy. 

%Studies that advocate for deep engagement tend to argue that the benefits of U.S. alliances are large and the costs are small. Our analysis does not challenge the former but indicates that alliances are more expensive than the deep engagement camp acknowledges. 

%Are these (and other) benefits sufficient to justify a price tag of \$11-21 billion per alliance? Our study is not designed to answer this question, . It would be harder to argue that the U.S. alliance network is desirable if the cost Washington pays to maintain it exceeds the value of the benefits, even if there are some clear positive effects produced by its security umbrella. By contrast, evidence that the benefits exceed this price tag would strengthen the argument for maintaining alliances as a centerpiece of U.S. grand strategy. 



Our conclusions are subject to three additional caveats. First, our model cannot reliably estimate how sudden and drastic changes in the U.S. alliance portfolio would affect defense expenditures, as these scenarios extrapolate far beyond our observed data. The greatest number of new commitments the United States took on in a single year from 1947 to 2019 was eleven, and the largest annual decrease in alliance commitments is one. We do not know how shifts larger than these in one year would influence U.S. military spending because we never observed such a scenario. Changes larger than anything we observed might generate different and unexpected effects. Although we might want to know what would happen to the budget if the United States eliminated \textit{all} of its alliance commitments next year, our model cannot estimate this effect.

Second, there are only four years in which we observe a reduction in the number of countries with U.S. protection. Our results are therefore based mostly on what happens after the United States takes on new commitments. Much of the policy debate today is about eliminating or curtailing existing alliance promises -- not taking on new ones. Our calculations can speak to this debate if we assume that forming new alliances and eliminating old ones are two sides of the same coin. However, if voiding an alliance has different implications for the U.S. defense budget than not forming one in the first place, our model would have less utility for the contemporary debate. To illustrate, some have argued that withdrawing from alliances would embolden U.S. adversaries and ultimately necessitate a large arms buildup to make up for lost ground.\autocite{colbyTNI16} If this claim is true, and if adversaries would \textit{not} have been more assertive had the United States refrained from forming an alliance at the outset, our analysis overestimates the savings the United States would get by eliminating alliance commitments.

Third, the factors that made alliances expensive for the United States over the last 70 years could change in the future. The United States expended considerable resources on policy coordination, military equipment, troops, overseas bases, and military exercises to deter adversaries and reassure allies. %Washington also had to compensate for inadequate burden sharing on the part of allies. 
Moving forward, improvements in the international security environment or changing beliefs among U.S. officials about how to make an effective alliance could reduce the direct costs of security guarantees. Washington could also save money in the future if allies make larger contributions towards common goals. This underscores an important reality: It is not alliance treaties per se that generate costs but rather what the United States does to secure and support them.\autocite[84-85]{rapphoopershields20} U.S. officials who are concerned about the cost of security guarantees could make them more efficient rather than eliminate alliances altogether.

%This brings us to an important policy question: is the price of U.S. alliance commitments worth it? Our analysis weakens the argument that alliances are inexpensive for the United States. But this does not necessarily imply that forming and maintaining them is a bad idea. To know whether U.S. investments in military alliances are worthwhile, we need to understand the benefits the United States obtains from being part of them, not just the costs. 

%We have shown that alliances are more expensive for the United States than other studies would lead us to believe. This weakens the argument that alliances . But answering this question requires an understanding of the benefits the United States obtains from forming alliances, not just the costs. 

%Scholarship identifies several ways in which providing security guarantees enhances U.S. interests. Alliances make the U.S. military more effective by facilitating power projection and augmenting the capabilities it can bring to bear during crises and military conflicts.\autocite[22-25]{BrandsFeaver2017} Formal security guarantees also promote global international peace and stability by enhancing extended deterrence.\autocite{leedsAJPS03,JohnsonLeeds2011,FuhrmannSechser2014} In addition, alliances help limit a major threat to U.S. national security: the international spread of nuclear weapons. U.S. alliances are a powerful instrument of nonproliferation because they reduce the ally's need for an independent nuclear arsenal.\autocite{bleekJCR14,reiterFPA14} Having alliance partners may also enhance U.S. political and diplomatic influence, while simultaneously weakening the sway of its adversaries.\autocite{} On the economic front, alliances facilitate trade and attract foreign direct investment.\autocite{Gowa:1993aa,longJPR03,liJIBS10}

%Those who place a premium on these benefits might argue that they are sufficient to justify a price tag of \$10-24 billion per alliance annually. Ultimately, a full accounting of this issue is beyond the scope of our analysis. By providing a more complete estimate of the financial cost of alliances, however, our study gives analysts and policymakers useful information for weighing the net benefits of alliance agreements for the United States.

%Whether this is the case, though, depends on how much one values the benefits of alliance commitments and the degree to which they materialize. Ultimately, a full accounting of this issue is beyond the scope of our analysis. By providing a more complete estimate of the financial cost of alliances, however, our study gives analysts and policymakers useful information for weighing the net benefits of alliance agreements for the United States.


